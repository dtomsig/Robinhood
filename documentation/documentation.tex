\documentclass{article}
    \usepackage[left = 1in, right = 1 in, top = 4 in]{geometry}
    \usepackage{hyperref}
    \usepackage{tcolorbox}
    \tcbset{colframe = white}
    \hypersetup
    {
        pdfborder            = {0 0 0},
        colorlinks           = true,
        linkcolor            = [RGB]{46 46 184},
        linktoc              = all,
        urlcolor             = blue,   
    }
    \title{\textbf{Robinhood Broker Transactions OFX and QIF File Generation}}
    \author{Diego Tomsig}
    \date{06-22-2018}
    \begin{document}
        \maketitle
        \newpage
        \newgeometry{left = 1 in, right = 1 in, top = 1 in, bottom = 1 in}
        \tableofcontents
        \newpage
        \section{Introduction}
            Generate.py produces two output files. The first is a QIF file 
            describing stock dividends, stock purchases, stock sales and short 
            and long term capital gains and losses for a given Robinhood 
            brokerage account. These transactions are those that occur on or 
            after an entered date. The reason why an entered date is required is
            for GnuCash importing and is described in that section.
            \newline
            \newline
            The second is an OFX file that describes all securities that are 
            encountered in the entire Robinhood history. This helps importing 
            the above file into GnuCash. It helps with the automatic creation of
            securities in GnuCash. It is suggested to import this first before 
            the QIF file. This is because the securities will be imported and 
            created before the transactions relating to those securities will 
            have occurred. This prevents the user from having to manually enter 
            security data if they were to enter transactions first.
            \newline
            \newline
            The overall purpose of this program is importing stock data from 
            Robinhood brokerage into GnuCash. These output files are located in
            the “output files” directory.
        \newpage
        \section{Installation \& Usage}
            The example presented was done on Debian 9. The steps are simple 
            enough to be done on any operating system.
            \newline
            \begin{enumerate}
                \item
                    Install python 3.3 or above.
                    \newline  
                    \begin{tcolorbox}[colback = cyan] 
                        sudo apt-get install python3
                    \end{tcolorbox}
                \item
                    Perform clone.
                    \newline  
                    \begin{tcolorbox}[colback = cyan] 
                        sudo git clone https://github.com/dtomsig/Robinhood.git
                    \end{tcolorbox}
                \item
                    Run program.
                    \newline  
                    \begin{tcolorbox}[colback = cyan] 
                        sudo python3 generate.py
                    \end{tcolorbox}
            \end{enumerate}
        \newpage
        \section{Importing Into GnuCash}
    \end{document}
